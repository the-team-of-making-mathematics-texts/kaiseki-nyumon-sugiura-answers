\documentclass[dvipdfmx,uplatex,11pt]{jsarticle}
%
\usepackage[dvipdfmx]{graphicx}
\usepackage{amsmath,amssymb,amsthm}
\usepackage{enumitem}
\usepackage{wrapfig}
\usepackage{bm}
\usepackage{ascmac}
\setcounter{tocdepth}{2}
\usepackage{ulem}
\usepackage{geometry}
\usepackage{framed}
\usepackage{latexsym}
%
\geometry{left=10mm,right=10mm,top=5mm,bottom=10mm}
%
\begin{document}
%
%
%
\section{第5章:級数}
\subsection{p49}
\noindent
問3:$\sum a_n$が絶対収束するとき,$\sum a_n ^2$も絶対収束することを証明せよ.
\\
\textsl{Hint}:級数の定義をおさえる\\
\dotfill

\begin{leftbar}
\begin{proof}

級数$\sum a_n$が絶対収束するから,$\sum |a_n|$も収束する.\\
また,$\sum a_n$が収束することから,$\displaystyle \lim_{n \to \infty} a_n=0$.
したがって,
\[
1>0に対して,あるn_0 \in \mathbb{N}が存在して,n \ge n_0 \Rightarrow |a_n|<1
\]
が成り立つ.\\
ここで,$|a_n|^2=|a_n ^2|$であり,\\
\[ |a_n ^2| <1 \]
が成り立つ.
したがって,$\sum |a_n ^2|$は収束するから,$\sum a_n ^2$は絶対収束する.
\end{proof}
\end{leftbar}
















\end{document}