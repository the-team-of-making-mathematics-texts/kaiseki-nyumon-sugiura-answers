\documentclass[dvipdfmx,uplatex,11pt]{jsarticle}
%
\usepackage[dvipdfmx]{graphicx}
\usepackage{amsmath,amssymb,amsthm}
\usepackage{physics}
\usepackage{enumitem}
\usepackage{wrapfig}
\usepackage{bm}
\usepackage{ascmac}
\setcounter{tocdepth}{2}
\usepackage{ulem}
\usepackage{siunitx}
\usepackage{geometry}
\usepackage{framed}
\usepackage{latexsym}
\everymath{\displaystyle}
%
\geometry{left=10mm,right=10mm,top=5mm,bottom=10mm}

\makeatletter
\renewcommand{\theequation}{
\thesection.\arabic{equation}}
\@addtoreset{equation}{section}

\renewcommand{\thefigure}{
\thesection.\arabic{figure}}
\@addtoreset{figure}{section}

\renewcommand{\thetable}{
\thesection.\arabic{table}}
\@addtoreset{table}{section}
\makeatother

\theoremstyle{definition}
\newtheorem{theo}{定理}[section]
\newtheorem{defi}{定義}[section]
\newtheorem{lemm}{補題}[section]
\newtheorem{lproof}{証明}[section]
\newtheorem{exam}{具体例}[section]
%
\renewcommand\proofname{\bf 証明}
%
\title{解析入門:解答集}
\author{編集者:}
\date{最終更新日:\today}
%
\begin{document}
\maketitle
\tableofcontents
\newpage

\section{実数と連続}

\subsection{p16,17}

\noindent
問2:
\\
\textsl{Hint}:$(n!x)$が整数となる条件を丁寧に調べていく.\\
\dotfill
%
% 
%
\\
$n=1,2,\ldots$に対して,
\[
	f_{n} (x)=\lim_{m \to \infty} (\cos (n! \pi x)) ^{2m}
\]
とおく.
ここで,$n!x \in \mathbb{Z}$のとき,
\[
	\cos (n! \pi x)=\pm 1
\]
$n!x \notin \mathbb{Z}$のときは,
\[
	|\cos (n! \pi x)|<1
\]
であるから,
\[
	f_{n} (x)=
	\begin{cases}
		1 \quad(x \in \mathbb{Z}) \\
		0 \quad (x \notin \mathbb{Z})
	\end{cases}
\]
となる.
さて,$x \in \mathbb{R} \backslash\mathbb{Q}$であるならば,どんな$n \in \mathbb{N}$に対しても,$n! x$が整数とならない.\\
$x \in \mathbb{Q}$のとき,$ x=\frac{p}{q}(p,q \in \mathbb{Z},q>0)$とすれば,$n$が$q$より十分大きく,$n \ge q$のとき,$n!x$は偶数.\\
よって,
\[
	\lim_{n \to \infty} \left( \lim_{m \to \infty} (\cos (n! \pi x)) ^{2m} \right)=
	\begin{cases}
		1 \quad (x \in \mathbb{Q}) \\
		0  \quad (x \in \mathbb{R} \backslash \mathbb{Q})
	\end{cases}
\]

\newpage

\noindent
問3:
\\
\textsl{Hint}:$a=0$のときを考えればよいので,そのように式変形をしてみる.そのあとは$\varepsilon$を用いて,評価していく。\\
\dotfill

\begin{leftbar}
	\begin{proof}
		$\lim_{n \to \infty} a_n= a$は,$ \lim_{n \to \infty} (a_n - a)= 0$\\
		と書き直せる.$b_n = a_n -a$とおくと,証明すべきことは,
		\[
			\lim_{n \to \infty} b_n=0,ならば \lim_{n \to \infty} \frac{b_1 + b_2 + \cdots +b_n}{n} = 0
		\]
		である.\par
		ここで,任意の$\varepsilon > 0$に対して,ある自然数$n_1 \in \mathbb{N}$が存在し,$n \ge n_1$のとき
		\[
			|b_n|<\varepsilon
		\]
		であり,また,絶対値の性質により,
		\[
			\left| \frac{b_1 + b_2 + \cdots +b_n}{n} \right| <\frac{|b_1| + |b_2| + \cdots +|b_n|}{n}
		\]
		をがいえる.このとき,$n \ge n_1$をみたす$(b_n)_{n \in \mathbb{N}}$の項を$\varepsilon$でおきかえると,
		\[
			\left| \frac{b_1 + b_2 + \cdots +b_n}{n} \right| < \frac{|b_1|+ \cdots + |b_{n_1}|}{n} +\varepsilon
		\]
		という不等式を得る.\par
		そこで,$n_2 \in \mathbb{N}$をとると,$n \ge n_2$のとき,
		\[
			\frac{|b_1|+ \cdots + |b_{n_1}|}{n} <\varepsilon
		\]
		であるとすると,$n \ge \max \{ n_1,n_2\}$のとき,
		\[
			\frac{|b_1|+ \cdots + |b_{n_1}|}{n} < \varepsilon
		\]
		となる.
		ゆえに,このとき,
		\[
			\left| \frac{ b_1 + b_2 + \cdots +b_n}{n} \right| < \varepsilon+\varepsilon=2\varepsilon
		\]
		よって示された.
	\end{proof}
\end{leftbar}

\newpage

\setcounter{equation}{0}
\subsection{p16}
\noindent
問5:
\\
\textsl{Hint}:\\
\dotfill
%
% 
%
\begin{leftbar}
	\begin{proof}
		$\mathbb{N} \ni m \ge 1$とする.$A \subset \mathbb{N}$が,与えられた条件を満たすとする.\par
		イ)より,
		\[ A \subset \{n \in \mathbb{N} \mid n \ge m\} \]
		は明らか.\\
		$H=\{0,1,\cdots,m-1\} \cup A$とおく.
		\begin{equation}
			0 \in H
		\end{equation}
		\begin{equation}
			n \in H
		\end{equation}
		とする.\\
		\[ n<m-1であれば,n+1 \in \{0,1,\cdots,m-1\}~より,n+1 \in H \]
		\[ n=m-1であれば,n+1=m \in Aより,n+1 \in H\]
		\[ n \ge mであれば,ロ)より,n+1 \in A \subset H \]
		よって,
		\[
			n+1 \in H
		\]
		したがって,$H$は継承的であり,$\mathbb{N} \subset H$\\
		つまり,\\
		\[
			\{0,1,\cdots,m-1\} \cup \{n \in \mathbb{N} \mid n \ge m \} \subset \{0,1,\cdots,m-1\} \cup A
		\]
		よって,
		\[
			\{n \in \mathbb{N}\mid n \ge m \} \subset A
		\]
		以上より,
		\[
			A=\{n \in \mathbb{N} \mid n \ge m\}
		\]
	\end{proof}
\end{leftbar}

\newpage

\noindent
問7:
\\
\textsl{Hint}:\\
\dotfill

\begin{leftbar}
	\begin{proof}
		$n$が自然数ならば,$n <k <n+1$となる自然数$k$は存在しないことを示す.\\
		$n < k < n+1$となる自然数$k$が存在するとすると,辺々$n$を引いて,
		\[ 0 < k - n < 1 \]
		問6より,$k-n$は自然数である.よって,$0<a<1$となる自然数$a$が存在しないことを示せばよい.\\
		\[ H=\{0\} \cup \{n \in \mathbb{N} \mid n \ge 1 \} \]
		とおくと,
		\begin{equation}
			0 \in H
		\end{equation}
		\begin{equation}
			0+1 \in H
		\end{equation}
		である.\\
		また,$k \ge 1$となる$k \in \mathbb{N}$に対しては,$k +1 \ge 1$であり,$k+1 \in \{~n \in \mathbb{N}~|~n \ge 1 \} $となるため,$k+1 \in H$\\
		よって$H$は継承的である.したがって,$\mathbb{N} \in H$\\
		$H$の定め方より,$a \notin H$なので,$a \notin \mathbb{N}$となり,$0<a<1$となる自然数$a$は存在しない.
	\end{proof}
\end{leftbar}

\newpage

\setcounter{equation}{0}
\subsection{p31}
\noindent
問2:
\\
\textsl{Hint}:\\
\dotfill
​
\begin{leftbar}
	\begin{proof}
		二項定理を用いて$(a_n)_{n \in \mathbb{N}}$の一般項を展開すると,
		\begin{eqnarray}
			a_n & = & 1 + n \cdot \frac{1}{n} + \frac{n(n-1)}{2!} \cdot \frac{1}{n^2} + \cdots + \frac{n(n-1)\cdots(n-r+1)}{r!} \cdot \frac{1}{n^r} + \cdots \frac{n!}{n!} \cdot + \frac{1}{n^n} \nonumber  \\
			& = & 1+ \frac{1}{1!} + \frac{1}{2!} \left(1- \frac{1}{n} \right) + \cdots + \frac{1}{r!} \cdot  \left(1 - \frac{1}{n} \right) \cdots \left (1-\frac{r-1}{n} \right) + \cdots \nonumber +  \frac{1}{n!} \left(1 - \frac{1}{n} \right) \cdots \left(1- \frac{n-1}{n} \right) \label{p32.問2:1}
		\end{eqnarray}
		同様にして,$a_{n+1}$の展開式を得たとき,$ \frac{1}{n+1} < \frac{1}{n}$より,$r\in \{ 1,2,\cdots ,n\}$に対して,
		\begin{equation}
			\frac{1}{r!} \cdot  \left(1 - \frac{1}{n} \right) \cdots \left (1-\frac{r-1}{n} \right) < \frac{1}{r!} \cdot  \left(1 - \frac{1}{n+1} \right) \cdots \left (1-\frac{r-1}{n+1} \right)
		\end{equation}
		が成立する.これと,$a_{n+1}$の展開式のほうが,正の項を一つ多く含むことから,
		\begin{equation}
			a_{n} < a_{n+1} \quad (\forall n \ge 1)
		\end{equation}
		が成立し,$(a_n)_{n \in \mathbb{N}}$は単調増加.また,\eqref{p32.問2:1}より,
		\begin{eqnarray}
			\label{p32.問2:2}
			a_n &<& 1 + \frac{1}{1!} + \frac{1}{2!} + \cdots + \frac{1}{n!} \\
			& < & 1 + \frac{1}{1!} + \frac{1}{2^2} + \cdots + \frac{1}{2^n} \\
			& < & 2 + \frac{1}{2} \left( \frac{ 1 - 2^{1-n} }{ 1- \frac{1}{2} } \right)\\
			& = & 2+ \frac{2^{1-n} -1}{2} \\
			\label{p32.問2:3}
			& < & 3
		\end{eqnarray}
		であるから,$a_n < 3$であり,また,
		\begin{equation}
			\label{p32.問2:4}
			a_n > 1 + \frac{1}{1!} =2
		\end{equation}
		であるから,\eqref{p32.問2:3},\eqref{p32.問2:4}より,$2<e<3$\\
		また,\eqref{p32.問2:2}より,
		\begin{equation}
			\label{p32.問2:5}
			\lim_{n \to \infty} a_n \le e
		\end{equation}
		であり,また,\eqref{p32.問2:1}より,
		\begin{equation}
			\label{p32.問2:6}
			\lim_{n \to \infty} a_n \le a_n < 1 + \frac{1}{1!} + \frac{1}{2!} + \cdots + \frac{1}{r!}
		\end{equation}
		\eqref{p32.問2:6}より,$r \to \infty$として,
		\begin{equation}
			\label{p32.問2:7}
			\lim_{n \to \infty} a_n \ge e
		\end{equation}
		となり,\eqref{p32.問2:5},\eqref{p32.問2:7}より,$\lim_{n \to \infty} a_n =e$を得る.
	\end{proof}
\end{leftbar}
​
\subsection{p72,73,74}


​
問4

\newpage

\section{微分法}

\newpage

\section{初等函数}

\newpage

\section{積分法}

\newpage

\section{級数}

\end{document}

