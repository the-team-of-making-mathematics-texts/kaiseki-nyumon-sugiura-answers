\documentclass[dvipdfmx,uplatex,11pt]{jsarticle}
%
\usepackage[dvipdfmx]{graphicx}
\usepackage{amsmath,amssymb,amsthm}
\usepackage{enumitem}
\usepackage{wrapfig}
\usepackage{bm}
\usepackage{ascmac}
\setcounter{tocdepth}{2}
\usepackage{geometry}
\usepackage{framed}
\usepackage{latexsym}
\everymath{\displaystyle}

\theoremstyle{definition}
\newtheorem{theo}{定理}[section]
\newtheorem{prop}{命題}[section]
\newtheorem{defi}{定義}[section]
\newtheorem{lemm}{補題}[section]
\newtheorem{exam}{具体例}[section]
%
\geometry{left=10mm,right=10mm,top=5mm,bottom=10mm}
%
\title{解析入門:解答集}
\author{編集者:}
\date{最終更新日:\today}
%
\begin{document}
\maketitle
\tableofcontents
\newpage
%
\section{実数と連続}
%
\subsection{p2}
%
問1;\\
\textsl{Hint}:\\
\dotfill
%
\\
\begin{itembox}[c]{(1)}
    \begin{proof}
$0,0' \in K$がともに加法単位元の性質を満たすとする.\par 
このとき,$0$が加法単位元の性質をもつことから,
\[
    0'+0=0'
\]
同様に,$0'$が加法単位元の性質をもつことから,
\[
    0+0' = 0
\]
交換律より,$0+0'=0'+0$なので,
\[
    0'=0'+0 =0+0' =0
\]
これからただちに加法単位元の一意性が従う.
    \end{proof}
    \end{itembox}
    \begin{itembox}[c]{(2)}
        \begin{proof}
$a ,b \in K$とし,
\[
    a+b =0
\]
とする.このとき,
\[
    -a = -a+0 = -a +(a+b)=(-a+a)+b =0+b = b
\]
となり,加法逆元の一意性が従う.
        \end{proof}
    \end{itembox}

    \begin{itembox}[c]{(3)}
        \begin{proof}
$a \in K$のとき,
\begin{gather*}
    a+(-a)=0 \\
    \therefore (-a)+a =0
\end{gather*}
他方,$-(-a)$は$(-a)$の加法逆元であるから,
\[
    (-a)+(-(-a))=0
\]
これと逆元の一意性により,$a=-(-a)$が従う.
\end{proof}
\end{itembox}
\begin{itembox}[c]{(5)}
    \begin{proof}
    $a \in K$に対して,
    \[
        a+(-1)a=(1+(-1))a =0a =0
    \]
    であるから,$-a$が$a$の加法逆元であることと含めて主張が従う.
    \end{proof}
\end{itembox}
\newpage 
\begin{itembox}[c]{(6)}
    \begin{proof}
    (4)の結果を用いる.$a=-1$とすると,
    \[
        (-1)(-1)=-(-1)=1
    \]
    これが証明すべきことであった.
    \end{proof}
\end{itembox}
\begin{itembox}[c]{(7)}
    \begin{proof}
    $a,b \in K$に対して,
    \begin{align*}
     a(-b)+ab & = a((-b)+b) \\
     & = a0 \\
    & =0 \\
\therefore \quad & a(-b)=-ab 
    \end{align*}
$(-a)b = -ab$も同様にして示される.
\end{proof}
\end{itembox}
%
\newpage
\subsection{p16,17}
%
\noindent
問2:
\\
\textsl{Hint}:$(n!x)$が整数となる条件を丁寧に調べていく.\\
\dotfill
%
% 
%
\\
$n=1,2,\ldots$に対して,
\[
	f_{n} (x)=\lim_{m \to \infty} (\cos (n! \pi x)) ^{2m}
\]
とおく.
ここで,$n!x \in \mathbb{Z}$のとき,
\[
	\cos (n! \pi x)=\pm 1
\]
$n!x \notin \mathbb{Z}$のときは,
\[
	|\cos (n! \pi x)|<1
\]
であるから,
\[
	f_{n} (x)=
	\begin{cases}
		1 \quad(n!x \in \mathbb{Z}) \\
		0 \quad (n!x \notin \mathbb{Z})
	\end{cases}
\]
となる.
さて,$x \in \mathbb{R} \backslash\mathbb{Q}$であるならば,どんな$n \in \mathbb{N}$に対しても,$n! x$が整数とならない.\\
$x \in \mathbb{Q}$のとき,$ x=\frac{p}{q}(p,q \in \mathbb{Z},q>0)$とすれば,$n$が$q$より十分大きく,$n \ge q$のとき,$n!x$は整数.\\
よって,
\[
	\lim_{n \to \infty} \left( \lim_{m \to \infty} (\cos (n! \pi x)) ^{2m} \right)=
	\begin{cases}
		1 \quad (x \in \mathbb{Q}) \\
		0  \quad (x \in \mathbb{R} \backslash \mathbb{Q})
	\end{cases}
\]
%
\newpage
%
\noindent
問3:
\\
\textsl{Hint}:$a=0$のときを考えればよいので,そのように式変形をしてみる.そのあとは$\varepsilon$を用いて,評価していく.\\
\dotfill
%
\begin{leftbar}
	\begin{proof}
		$\lim_{n \to \infty} a_n= a$は,$ \lim_{n \to \infty} (a_n - a)= 0$\\
		と書き直せる.$b_n = a_n -a$とおくと,証明すべきことは,
		\[
			\lim_{n \to \infty} b_n=0,ならば \lim_{n \to \infty} \frac{b_1 + b_2 + \cdots +b_n}{n} = 0
		\]
		である.\par
		ここで,任意の$\varepsilon > 0$に対して,ある自然数$n_1 \in \mathbb{N}$が存在し,$n \ge n_1$のとき
		\[
			|b_n|<\varepsilon
		\]
		であり,また,絶対値の性質により,
		\[
			\left| \frac{b_1 + b_2 + \cdots +b_n}{n} \right| <\frac{|b_1| + |b_2| + \cdots +|b_n|}{n}
		\]
		がいえる.このとき,$n \ge n_1$をみたす$(b_n)_{n \in \mathbb{N}}$の項を$\varepsilon$でおきかえると,
		\[
			\left| \frac{b_1 + b_2 + \cdots +b_n}{n} \right| < \frac{|b_1|+ \cdots + |b_{n_1}|}{n} +\varepsilon
		\]
		という不等式を得る.\par
		そこで,$n_2 \in \mathbb{N}$をとると,$n \ge n_2$のとき,
		\[
			\frac{|b_1|+ \cdots + |b_{n_1}|}{n} <\varepsilon
		\]
		であるとすると,$n \ge \max \{ n_1,n_2\}$のとき,
		\[
			\frac{|b_1|+ \cdots + |b_{n_1}|}{n} < \varepsilon
		\]
		となる.
		ゆえに,このとき,
		\[
			\left| \frac{ b_1 + b_2 + \cdots +b_n}{n} \right| < \varepsilon+\varepsilon=2\varepsilon
		\]
		よって示された.
	\end{proof}
\end{leftbar}
%
\newpage
%
\setcounter{equation}{0}
\noindent
問5:
\\
\textsl{Hint}:\\
\dotfill
%
% 
%
\begin{leftbar}
	\begin{proof}
		$\mathbb{N} \ni m \ge 1$とする.$A \subset \mathbb{N}$が,与えられた条件を満たすとする.\par
		イ)より,
		\[ A \subset \{n \in \mathbb{N} \mid n \ge m\} \]
		は明らか.\\
		$H=\{0,1,\cdots,m-1\} \cup A$とおく.
		\begin{equation}
			0 \in H
		\end{equation}
		\begin{equation}
			n \in H
		\end{equation}
		とする.\\
		\[ n<m-1であれば,n+1 \in \{0,1,\cdots,m-1\}~より,n+1 \in H \]
		\[ n=m-1であれば,n+1=m \in Aより,n+1 \in H\]
		\[ n \ge mであれば,ロ)より,n+1 \in A \subset H \]
		よって,
		\[
			n+1 \in H
		\]
		したがって,$H$は継承的であり,$\mathbb{N} \subset H$\\
		つまり,\\
		\[
			\{0,1,\cdots,m-1\} \cup \{n \in \mathbb{N} \mid n \ge m \} \subset \{0,1,\cdots,m-1\} \cup A
		\]
		よって,
		\[
			\{n \in \mathbb{N}\mid n \ge m \} \subset A
		\]
		以上より,
		\[
			A=\{n \in \mathbb{N} \mid n \ge m\}
		\]
	\end{proof}
\end{leftbar}
%
\newpage
%
\noindent
問7:
\\
\textsl{Hint}:\\
\dotfill
%
\begin{leftbar}
	\begin{proof}
		$n$が自然数ならば,$n <k <n+1$となる自然数$k$は存在しないことを示す.\\
		$n < k < n+1$となる自然数$k$が存在するとすると,辺々$n$を引いて,
		\[ 0 < k - n < 1 \]
		問6より,$k-n$は自然数である.よって,$0<a<1$となる自然数$a$が存在しないことを示せばよい.\\
		\[ H=\{0\} \cup \{n \in \mathbb{N} \mid n \ge 1 \} \]
		とおくと,
		\begin{equation}
			0 \in H
		\end{equation}
		\begin{equation}
			0+1 \in H
		\end{equation}
		である.\\
		また,$k \ge 1$となる$k \in \mathbb{N}$に対しては,$k +1 \ge 1$であり,$k+1 \in \{~n \in \mathbb{N}~|~n \ge 1 \} $となるため,$k+1 \in H$\\
		よって$H$は継承的である.したがって,$\mathbb{N} \subset H$\\
		$H$の定め方より,$a \notin H$なので,$a \notin \mathbb{N}$となり,$0<a<1$となる自然数$a$は存在しない.
	\end{proof}
\end{leftbar}
%
\newpage
%
\setcounter{equation}{0}
\subsection{p31}
問1:\\
(i)
\begin{leftbar}
	\begin{align*}
		\frac{1^2+2^2+\cdots+n^2}{n^3} & = \frac{\dfrac{1}{6}n(n+1)(2n+1)}{n^3} \\
		& =\frac{1}{6} \left(1+\frac{1}{n} \right ) \left(2+\frac{1}{n} \right)
	\end{align*}
	定理2.5(2)より,
	\[
		\frac{1}{6} \lim_{n \to \infty} \left(1+\frac{1}{n} \right) \cdot \lim_{n \to \infty} \left(2+\frac{1}{n} \right) =\lim_{n \to \infty} \frac{1}{6} \left(1+\frac{1}{n} \right ) \left(2+\frac{1}{n} \right)
	\]
	である.また,アルキメデスの原理により, 任意の$\varepsilon >0$に対して,$n_0 \in \mathbb{N}$を$n_0 >\frac{1}{\varepsilon}$となるようにとることができ,このとき,任意の$n \in \mathbb{N}$に対して,
	\[
		n \ge n_0 \Longrightarrow 0<\frac{1}{n} \le \frac{1}{n_0} = \varepsilon
	\]
	となり,$\lim_{n \to \infty} \frac{1}{n} =0$である.これより,
	\begin{align*}
		\lim_{n \to \infty} \frac{1^2+2^2+\cdots+n^2}{n^3} & = \lim_{n \to \infty} \frac{1}{6} \left(1+\frac{1}{n} \right ) \left(2+\frac{1}{n} \right) \\
		& = \frac{1}{6} (1+0)(2+0) =\frac{1}{3}
	\end{align*}
\end{leftbar}
\newpage
(ii)
\begin{lemm}
    \label{p31:問1.2}
    正数列$(a_n)_{n \in \mathbb{N}}$に対して,$\left(\frac{a_{n+1}}{a_n} \right)_{n \in \mathbb{N}}$が収束し,
    \[
    \lim_{n \to \infty} \frac{a_{n+1}}{a_n} <1
    \]
    となるとき,$\lim_{n \to \infty} a_n =0$である.
\end{lemm}
%
\begin{proof}(補題\ref{p31:問1.2}の証明)\par 
   $ \lim_{n \to \infty} \frac{a_{n+1}}{a_n} <1$であるから,$r~(0<r<1)$に対して,ある$N_1 \in \mathbb{N}$が存在して,任意の$n \in \mathbb{N}$に対して,
   \[
       n \ge N_1 \Longrightarrow \frac{a_{n+1}}{a_n}<r
   \]
   が成り立つ.このとき,
   \[
       a_n = a_{N_1} \cdot \frac{a_{N_1+1}}{a_{N_1}} \cdot \frac{a_{N_1 +2}}{a_{N_1 +1}} \cdots \frac{a_{n-1}}{a_{n-2}} \frac{a_n}{a_{n-1}}< a_{N_1} r^{n-N_1}=\frac{a_{N_1}}{r^{N_1}} r^n
   \]
   となる.$0<r<1$より$\lim_{n \to \infty} \frac{a_{N_1}}{r^{N_1}} r^n =0$であるから,$\lim_{n \to \infty} a_n =0$である.
\end{proof}
%
\begin{leftbar}
        $a_n = \frac{n^2}{a^n}$とおく.$0<a<1$のときは明らかに$\lim_{n \to \infty} a_n=\infty$となる.\par 
        $a>1$のとき,$\frac{a_{n+1}}{a_n} =\frac{1+\dfrac{1}{n}}{a}$となり,
        \[
            \lim_{n \to \infty} \frac{1+\dfrac{1}{n}}{a} = \frac{1}{a} <1
        \]
        であるから,補題\ref{p31:問1.2}により,$\lim_{n \to \infty} a_n =0$となる.以上の議論により,
        \begin{align*}
            \lim_{n \to \infty} \frac{n^2}{a^n}
            =
            \begin{cases}
                \infty & (0<a<1) \\
                0 & (a>1)
            \end{cases}
        \end{align*}
        となる.
	\end{leftbar}
	\newpage
	(iii)
\begin{leftbar}
    明らかに$\sqrt[n]{n} >1$なので,$\delta_n >0$を用いて,
    \[
        \sqrt[n]{n} = 1+ \delta_n
    \]
    とかける.両辺を$n$乗して,$n \to \infty$の極限を考えることを考慮すると,
    \begin{align*}
        n = (1+\delta_n)^n &=1 + n \delta_n + \frac{1}{2}n(n-1) {\delta_n}^2 + \cdots + (\delta_n)^n \\
        & > \frac{1}{2}n(n-1) {\delta_n}^2
    \end{align*}
    となり,この不等式から,
    \[
        (0<) ~\delta_n < \sqrt{\frac{2}{n-1}}
    \]
    を得る.ここで,はさみうちの原理により,$\lim_{n \to \infty} \delta_n =0$であるから,
    \[
        \lim_{n \to \infty} \sqrt[n]{n} =1
    \]
    である.
\end{leftbar}
\newpage
(iv)
\begin{leftbar}
      $a_n= n^k e^{-n}$とおく.このとき,
      \[
          \lim_{n \to \infty} \frac{a_{n+1}}{a_n} =  \lim_{n \to \infty} \frac{\left(1+\dfrac{1}{n}\right)^k}{e} =\frac{1}{e} <1
      \]
      ゆえに,補題\ref{p31:問1.2}により,
      \[
        \lim_{n \to \infty} n^k e^{-n}=0
      \]
      である.
\end{leftbar}
\newpage
(v)
\begin{leftbar}
      $a_n =\left (1-\frac{1}{n^2}\right)^n$とおく.
      \begin{align*}
        \lim_{n \to \infty} a_n & =\lim_{n \to \infty} \left (1-\frac{1}{n^2}\right)^n \\
        & = \lim_{n \to \infty} \left (1+\frac{1}{n}\right)^n \left (1-\frac{1}{n}\right)^n \\
        & = e \cdot \frac{1}{e} =1
      \end{align*}
      である.
\end{leftbar}
\newpage
(vi)
\begin{lemm}
    \label{p31:問1.6.1}
    $(a_n)_{n \in \mathbb{N}}$を実数列とし,$a_n \ne 0$とする.もし$\lim_{n \to \infty} a_n =0$であれば$\lim_{n \to \infty} \frac{1}{a_n}=\infty$である.
    また,もし$\lim_{n \to \infty} a_n =\infty$であるならば,$\lim_{n \to \infty} \frac{1}{a_n} =0$である,
\end{lemm}

\begin{proof}
    前半の主張のみ示せば後半の主張も同様に示せるので,前半のみ示す.与えられた条件により,任意の$\varepsilon>0$に対して,$n_0 \in \mathbb{N}$が存在して,任意の$n \in \mathbb{N}$に対して,
    \[
        n \ge n_0 \Longrightarrow |a_n-0|<\varepsilon
    \]
    が成り立つ,ここで$\frac{1}{\varepsilon}=M$とおくと,上の$n_0 \in \mathbb{N}$に対して,
    \[
        n \ge n_0 \Longrightarrow \left |\frac{1}{a_n} \right| >\frac{1}{\varepsilon}=M
    \]
    となり,これより$lim_{n \to \infty} \frac{1}{a_n}=\infty$が示された.
\end{proof}

\begin{lemm}
    \label{p31:問1.6.2}
    $c>1$のとき,$\lim_{n \to \infty} \frac{1}{c^n} = 0$である.
\end{lemm}

\begin{proof}
    $c>1$より,$\delta >0$を用いて$c=1+\delta$とおける.このとき,
    \begin{align*}
        c^n &= (1+\delta)^n
        & = 1+n \delta +\frac{1}{2}n (n-1) \delta^2 + \cdots +\delta^n
        & > 1+n \delta
    \end{align*}
    このことから,$0<\frac{1}{c^n} <\frac{1}{1+n\delta}$であるから,はさみうちの原理により,
    \[
        \lim_{n \to \infty} \frac{1}{c^n} = 0~(c>1)
    \]
    である.
\end{proof}

\begin{leftbar}
    $a_n = (c^n +c^{-n})^{-1}$とおく.$c=1$のときは明らかに$\lim_{n \to \infty} a_n =\frac{1}{2}$である.\par 
    $c>1$のとき,補題\ref{p31:問1.6.1},補題\ref{p31:問1.6.2}により,
    \[
        \lim_{n \to \infty} c^n + c^{-n} = \infty
    \]
    であるから,補題\ref{p31:問1.6.1}により$\lim_{n \to \infty} a_n = \lim_{n \to \infty} (c^n +c^{-n})^{-1} =0$である.\par 
    $0<c<1$のときは$c$の逆数を考えることにより同じ結論に帰着する.以上の議論により,
    \begin{align*}
        \lim_{n \to \infty} (c^n +c^{-n})^{-1} =
        \begin{cases}
            \frac{1}{2}&(c=1)\\
            0 & (c\ne 0,c>0)
        \end{cases}
    \end{align*}
    である.
\end{leftbar}
\newpage
\noindent
問2:\\
\textsl{Hint}:\\
\dotfill
\begin{leftbar}
	\begin{proof}
		二項定理を用いて$(a_n)_{n \in \mathbb{N}}$の一般項を展開すると,
		\begin{eqnarray}
			a_n & = & 1 + n \cdot \frac{1}{n} + \frac{n(n-1)}{2!} \cdot \frac{1}{n^2} + \cdots + \frac{n(n-1)\cdots(n-r+1)}{r!} \cdot \frac{1}{n^r} + \cdots \frac{n!}{n!} \cdot + \frac{1}{n^n} \nonumber  \\
			& = & 1+ \frac{1}{1!} + \frac{1}{2!} \left(1- \frac{1}{n} \right) + \cdots + \frac{1}{r!} \cdot  \left(1 - \frac{1}{n} \right) \cdots \left (1-\frac{r-1}{n} \right) + \cdots \nonumber +  \frac{1}{n!} \left(1 - \frac{1}{n} \right) \cdots \left(1- \frac{n-1}{n} \right) \label{p32 問2 1}
		\end{eqnarray}
		同様にして,$a_{n+1}$の展開式を得たとき,$ \frac{1}{n+1} < \frac{1}{n}$より,$r\in \{ 1,2,\cdots ,n\}$に対して,
		\begin{equation}
			\frac{1}{r!} \cdot  \left(1 - \frac{1}{n} \right) \cdots \left (1-\frac{r-1}{n} \right) < \frac{1}{r!} \cdot  \left(1 - \frac{1}{n+1} \right) \cdots \left (1-\frac{r-1}{n+1} \right)
		\end{equation}
		が成立する.これと,$a_{n+1}$の展開式のほうが,正の項を一つ多く含むことから,
		\begin{equation}
			a_{n} < a_{n+1} \quad (\forall n \ge 1)
		\end{equation}
		が成立し,$(a_n)_{n \in \mathbb{N}}$は単調増加.また,\eqref{p32 問2 1}より,
		\begin{eqnarray}
			\label{p32 問2 2}
			a_n &<& 1 + \frac{1}{1!} + \frac{1}{2!} + \cdots + \frac{1}{n!} \\
			& < & 1 + \frac{1}{1!} + \frac{1}{2^2} + \cdots + \frac{1}{2^n} \\
			& < & 2 + \frac{1}{2} \left( \frac{ 1 - 2^{1-n} }{ 1- \frac{1}{2} } \right)\\
			& = & 2+ \frac{2^{1-n} -1}{2} \\
			\label{p32 問2 3}
			& < & 3
		\end{eqnarray}
		であるから,$a_n < 3$であり,また,
		\begin{equation}
			\label{p32 問2 4}
			a_n > 1 + \frac{1}{1!} =2
		\end{equation}
		であるから,\eqref{p32 問2 3},\eqref{p32 問2 4}より,$2<e<3$\\
		また,\eqref{p32 問2 2}より,
		\begin{equation}
			\label{p32 問2 5}
			\lim_{n \to \infty} a_n \le e
		\end{equation}
		であり,また,\eqref{p32 問2 1}より,
		\begin{equation}
			\label{p32 問2 6}
			\lim_{n \to \infty} a_n \ge a_n > 1 + \frac{1}{1!} + \frac{1}{2!} + \cdots + \frac{1}{r!}
		\end{equation}
		\eqref{p32 問2 6}より,$r \to \infty$として,
		\begin{equation}
			\label{p32 問2 7}
			\lim_{n \to \infty} a_n \ge e
		\end{equation}
		となり,\eqref{p32 問2 5},\eqref{p32 問2 7}より,$\lim_{n \to \infty} a_n =e$を得る.
	\end{proof}
\end{leftbar}
\newpage
\subsection{p49~50}
問1:
\begin{leftbar}
	\begin{proof}
		$\lim_{n \to \infty} \sqrt[n]{a_n} =r$であるから,$0<r<1$のとき,$r<k<1$に対して,ある$n_1 \in \mathbb{N}$が存在して,任意の$n \in \mathbb{N}$に対して,
		\[
			n \ge n_1 \Longrightarrow a_n<k^n
		\]
		となる.ここで,定理5.5(比較判定法)と$\lim_{n \to \infty} k^n =0$より,$\sum a_n$は収束する.\par 
		$r>1$のとき,$1>0$に対して,ある$n_2 \in \mathbb{N}$が存在して,任意の$n \in \mathbb{N}$に対して,
		\[
			n \ge n_2 \Longrightarrow a_n >1
		\]
		が成り立ち,$\lim_{n \to \infty} a_n \ne 0$となる.よって定理5.1~系の対偶により$\sum a_n$は発散する.
	\end{proof}
\end{leftbar}

\newpage

\begin{screen}
	杉浦P49~2)(i)\\
	発散する.\\
	\dotfill \\
	{\it Proof.}
	\begin{eqnarray*}
	\frac{2n^2}{n^3+1}=\frac{2/3}{n+1}+\frac{4n/3-2/3}{n^2-n+1}
	\end{eqnarray*}
	により
	\begin{eqnarray*}
	\sum \frac{2n^2}{n^3+1}&=&\sum \frac{2/3}{n+1}+\sum \frac{4n/3-2/3}{n^2-n+1} \\
	&>&\sum \frac{2/3}{n+1} \rightarrow \infty
	\end{eqnarray*}
	\qed
	\end{screen}
	
	\begin{screen}
	(ii)収束する.\\
	\dotfill \\
	{\it Proof.}
	\begin{eqnarray*}
	\sum ^{\infty}_{n=1}\frac{\sqrt{n}}{1+n^2}<\sum \frac{\sqrt{n}}{n^2}=\sum^{\infty}_{n=1}\frac{1}{n^{3/2}}~\left(<1+\int^{\infty}_{1}\frac{dx}{x^{3/2}}=3\right)
	\end{eqnarray*}
	また
	\begin{eqnarray*}
	\sum \frac{1}{n^\alpha}
	\end{eqnarray*}
	が$\alpha >1$のときに収束することを用いることもできる.\qed
	\end{screen}
	\begin{screen}
	(iii)$a=1$のときは収束,$a \neq 1$のときは発散する.$\\
	\dotfill \\
	{\it Proof.}\\
	$a=1$のときは明らかに収束する.$\\
	$a \neq 1$のとき,$a^x$の$x=0$のまわりでの$\mathrm{Taylor}$展開
	\begin{eqnarray*}
	a^x=1+(\log a)x+\frac{(\log a)^2}{2!}x^2+O(x^3)
	\end{eqnarray*}
	を用いて
	\begin{eqnarray*}
	\sum (a^{1/n}-1)&=&\sum \left\{ \frac{\log a}{n}+\frac{(\log a)^2}{2!n^2}+O\left(\frac{1}{n^3}\right)\right\}\\
	&>&\sum \frac{\log a}{n} \rightarrow \infty\\
	\end{eqnarray*}
	\qed
	\end{screen}
	
	\begin{screen}
	(v)収束する.\\
	\dotfill \\
	{\it Proof.}\\
	定理5.7(ダランベールの収束判定)より$a_n=n/2^n$とおくと
	\begin{eqnarray*}
	\lim_{n \to \infty}\frac{a_{n+1}}{a_n}=\lim_{n \to \infty}\frac{n+1}{n}\frac{2^n}{2^{n+1}}=\frac{1}{2}<1
	\end{eqnarray*}
	よって収束する.
	\end{screen}
	
	\begin{screen}
	(viii)
	発散する.\\
	\dotfill \\
	{\it Proof.}
	\begin{eqnarray*}
	\frac{(1+n)^n}{n^{n+1}}>\frac{n^n}{n^{n+1}}=\frac{1}{n}
	\end{eqnarray*}
	より
	\begin{eqnarray*}
	\sum \frac{(1+n)^n}{n^{n+1}}>\sum \frac{1}{n} \rightarrow \infty
	\end{eqnarray*}
	よって発散する. \qed
	\end{screen}
	
	\begin{screen}
	(x)収束する.\\
	\dotfill \\
	{\it Proof.}\\
	定理5.7(ダランベールの収束判定)より$a_n=\left(\dfrac{n}{n+1}\right)^{n^2}$とすると
	\begin{eqnarray*}
	\frac{a_{n+1}}{a_n}&=&\frac{(n+1)^{(n+1)^2}}{(n+2)^{(n+1)^2}}\frac{(n+1)^{n^2}}{n^{n^2}}=\frac{(n+1)^{n^2}}{n^{n^2}}\frac{(n+1)^{n^2}}{(n+2)^{n^2}}\frac{(n+1)^{2n+1}}{(n+2)^{2n+1}}\\
	&=&\left(1+\frac{1}{n}\right)^{n^2}\left(1-\frac{1}{n+2}\right)^{n^2}\left(1-\frac{1}{n+2}\right)^{2n+1}
	\end{eqnarray*}
	ここで
	\begin{eqnarray*}
	\left(1+\frac{1}{n}\right)^{n^2}\left(1-\frac{1}{n+2}\right)^{n^2}=\left(1+\frac{1}{n(n+2)}\right)^{n^2}=\left(1+\frac{1}{n(n+2)}\right)^{n(n+2)}\left(1+\frac{1}{n(n+2)}\right)^{-2n}
	\end{eqnarray*}
	$とすることにより,右辺はe \times 1=eに収束.更に$
	\begin{eqnarray*}
	\left(1-\frac{1}{n+2}\right)^{2n+1}=\left\{\left(1-\frac{1}{n+2}\right)^{-(n+2)}\right\}^{-2}\left(1-\frac{1}{n+2}\right)^{-3}
	\end{eqnarray*}
	$とすることで,右辺は\dfrac{1}{e^2} \times 1=\dfrac{1}{e}に収束.よって$
	\begin{eqnarray*}
	\frac{a_{n+1}}{a_n}=e \times \frac{1}{e^2}=\frac{1}{e}<1
	\end{eqnarray*}
	よってこの級数は収束する.
\end{screen}

​\newpage

(3)
\begin{leftbar}
	\begin{proof}
		$\sum a_n $が絶対収束するので,$\sum |a_n|$も収束する.
		よって,定理5.1 系より,$\lim_{n \to \infty} |a_n| =0$となる.
		このとき,$1>0$に対して,ある$n_1 \in \mathbb{N}$が存在して,任意の$n \in \mathbb{N}$に対して,
		\[
			n \ge n_1 \Longrightarrow ||a_n| -0|<1
		\]
		が成り立つ.また,$|a_n|<1$のとき,
		\[
			0 \le |{a_n}^2| \le {|a_n|}^2 < |a_n|
		\]
		が成り立つ.ここで,$\sum |a_n|$が収束し,各項は正なので, 定理5.5(比較判定法)により,$\sum |{a_n}^2|$も収束する.ゆえに$\sum {a_n}^2$は絶対収束する.
	\end{proof}
\end{leftbar}
\newpage
(6)
\begin{lemm}(ライプニッツの定理) \par 
    \label{ライプニッツの定理}
    単調減少な正数列$(a_n)_{n \in \mathbb{N}}$について,$\lim_{n \to \infty} a_n =0$ならば,
    \[
        \sum_{n=0}^{\infty} (-1)^n a_n 
    \]
    は収束する.
\end{lemm}
\begin{proof}
    第$n$部分和を$s_n$とおく.つまり$s_n = \sum_{k=0}^{n} (-1)^k a_k$である.このとき,
    \[
        S_{2n-1} = (a_0 -a_1) + (a_2-a_3)+ \cdots +(a_{2n-2}-a_{2n-1})
    \]
    であり,$ a_{2k}-a_{2k-1} \le 0$であるから,数列$(s_{2n-1})_{n \in \mathbb{N}}$は単調増加である.また,
    \[
        S_{2n-1} = a_0 -(a_1- a_2)-(a_3-a_4)- \cdots -(a_{2n-3}-a_{2n-2})-a_{2n-1} \le a_{0}
    \]
    であるから,$(s_{2n-1})_{n \in \mathbb{N}}$は上に有界である.よって,$(s_{2n-1})_{n \in \mathbb{N}}$は収束する.ここで,
    \[
        \lim_{n \to \infty} s_{2n-1} =S
    \]
    とすると,
    \[
        \lim_{n \to \infty} s_{2n} =\lim_{n \to \infty} (s_{2n-1}+a_{2n})=S+0=S
    \]
    であるから,$(s_{2n})_{n \in \mathbb{N}}$も$S$に収束する.以上の議論により,$(s_n)_{n \in \mathbb{N}}$は収束する.
\end{proof}
\begin{leftbar}
    補題\ref{ライプニッツの定理}と,$\lim_{n \to \infty} \frac{1}{2n+1} =0$により,$\sum_{n=0}^{\infty} \frac{(-1)^n}{2n+1}$は収束する.ここで,
    \[
         1-x^2+x^4-\cdots+(-1)^n x^{2n} =\frac{1}{1+x^2} +\frac{(-1)^n x^{2n+2}}{x^2+1}
    \]
の両辺を0から1まで$x$で積分すると,
\begin{align*}
   \overbrace{1-\frac{1}{3}+\frac{1}{5}-\cdots+\frac{(-1)^n}{2n+1}}^{s_n} &=\int_{0}^{1} \frac{1}{1+x^2} \, dx +\int_{0}^{1}\frac{(-1)^n x^{2n+2}}{x^2+1}  \, dx \\
& = \frac{\pi}{4} + R_n
\end{align*}
である.ただしここで$R_n =\int_{0}^{1}\frac{(-1)^n x^{2n+2}}{x^2+1} \, dx$とおいた.この式から,
\begin{align*}
   \left |s_n -\frac{\pi}{4} \right | & = \left|\int_{0}^{1}\frac{(-1)^n x^{2n+2}}{x^2+1} \, dx \right| \\
   & < \int_{0}^{1} x^{2n} \, dx \\
   & =\frac{1}{2n+1} \to 0 ~(n \to \infty)
\end{align*}
である.よって,
\[
    \sum_{n=0}^{\infty} \frac{(-1)^n}{2n+1} =\frac{\pi}{4}
\]
である.
\end{leftbar}
\newpage
\subsection{p72,73,74}
問4:\par 
まず,二つの補題を証明する.\\
\begin{lemm}
	\label{p72:問4.1}
	ノルムに関して, $| \| \bm x  \|- \| \bm y \| | \leq \| \bm x - \bm y \|$ が成り立つ. 
\end{lemm}
\begin{proof} (補題\ref{p72:問4.1}の証明)\par 
	絶対値の定義 ( $| a | : = \max \{ a , -a \}$ ) に立ち返ると, 
%		
		\[
			\| \bm x  \|- \| \bm y \| \leq \| \bm x - \bm y \| , - \| \bm x  \| + \| \bm y \| \leq \| \bm x - \bm y \|
		\]
%		
	なる二つの不等式を示せばよい. これは, 
		
		\[
			\| \bm x \| = \| \bm x - \bm y + \bm y \| \leq \| \bm x - \bm y \| + \| \bm y \|
		\]
		
	より示される.
\end{proof}
ふたつめの補題を証明する.
\begin{lemm}
	\label{p72:問4.2}
	$\| \cdot \|_1 : \mathbb R^n \ni ( x_1 , x_2 , \cdots , x_n ) \mapsto \sum_{j=1} ^ n |x_j| \in \mathbb R$ とすると, 
%		
		\[
			\exists M \in \mathbb R \ \mathrm{s.t.} \ \forall \bm x \in \mathbb R^n \text {に対して, } \| \bm x \|_1 \leq M | \bm x | 
		\]
%		
	が成り立つ. 
\end{lemm}
\begin{proof} (補題\ref{p72:問4.2}の証明)\par 
	
	$M : = n$ とおく. $\bm x \in \mathbb R^n$ とする. 各 $j ( 1 \leq j \leq n )$ において, 
%		
		\[
			| x_j | \leq | \bm x |
		\]
%		
	が成り立つ. そこで, $j$ について足し合わせると, 
%		
		\[
			| x_1 | + | x_2 | + \cdots | x_n | \leq | \bm x | + | \bm x | + \cdots | \bm x | = n | \bm x | = M | \bm x | 
		\]
%		
	が成り立つ. 
\end{proof}
\dotfill
\begin{leftbar}
\begin{proof}
	まず, 
%		
		\[
			\exists P \in \mathbb R \ \mathrm{s.t.} \ \forall \bm x \in \mathbb R^n \text {に対して, } \| \bm x \| \leq P | \bm x | \ \ \ \cdots ( \ast )
		\]
%		
	となることを示す. $\mathbb R^n$ の正規直交基底 を $\bm e_1 , \bm e_2 , \cdots , \bm e_n$ とする. $T : = \max \{ \| \bm e_j \| \mid 1 \leq j \leq n \}$ とおき, $M$ を補題のものとして, $P : = TM$ とおく. . $\bm x \in \mathbb R ^n$ を, 
%		
		\[
			\bm x = ( x_1 , x_2 , \cdots , x_n )
		\]
%		
	とすると, $\bm x = \sum_{j=1} ^ n x_j \bm e_j$ とかける. これと, ノルムのi) , ii) の条件から, 
%		
		\[
			\| \bm x \| = \| \sum_{j=1} ^ n x_j \bm e_j \| \leq \sum | x_j | \| \bm e_j \| \leq T \sum | x_j | \leq T M | \bm x | = P | \bm x |
		\]
%		
	が成り立つ. 以上より $( \ast )$ は成り立つ. これより, 函数 $\| \cdot \|$ が 連続函数になる事を示す. $\bm \alpha \in \mathbb R^n$ とする. $\varepsilon > 0$ とする. $\delta < \frac {\varepsilon} {P}$ を満たすようにとる. このとき, 
	$\bm x \in \mathbb R^n$ かつ $| \bm x - \bm \alpha | < \delta$ とする. 
%		
		\[
			| \| \bm x \| - \| \bm \alpha \| | \leq \| \bm x - \bm \alpha \| \leq P | \bm x - \bm \alpha | < \varepsilon
		\] 
%		
	よって, 連続である.  $\mathbb S^{n-1} : =\{ \bm x \in \mathbb R^n \mid | \bm x | = 1 \}$ なるコンパクト集合上で, $\| \cdot \|$ を考える. ノルムが連続であったことから, コンパクト集合上で最小値をとる. これを $m$ とする. この 
	$m$ を用いると, 
%		
		\[
			\forall x \in \mathbb R^n \text {に対して, } m | \bm x | \leq \| \bm x \|
		\]
%		
	が成り立つ事を示す. $\bm x \in \mathbb R^n$ とする. このとき, $\bm x / | \bm x | \in \mathbb S^{n-1}$ であることに注意すると, ( $\because | \bm x / | \bm x | | = | \bm x | / | \bm x | = 1 $ ) $m$ が最小値であることから, 
%		
		\[
			m \leq \left \| \frac {\bm x} {| \bm x |} \right \|
		\]
%		
	を満たす. 
%		
		\[
			\left \| \frac {\bm x} {| \bm x |} \right \| = \frac {1} {| \bm x |} \| \bm x \|
		\]
%		
	に注意すると, 
%		
		\[
			m | \bm x | \leq \| \bm x \|
		\]
%		
	となる. 以上より, 示すべき題意は満たされた.	
\end{proof}
\end{leftbar}
これは, なかなか背景を語るには, 奥深い問題です. まず, 注意しておくことは, この性質は有限次元の線型空間だからできる話だということです. 無限次元であれば, もう少し条件がないと無理です. ( そもそも一般にはユークリッド距離が入りません. ) 有限次元であれば, ノルムが定める位相というのが一意に定まるという事を言っています. それだけ有限次元の線型空間は「硬い」という風に定義づけることができるでしょう. \par
次に証明の中で用いたテクニックです.  $\mathbb S^{n-1}$ を用いたところ. これは, 無限次元になっても用いられる手法です. 線型性がある操作や空間の中では, 綺麗な ( 空間全体に一様に広げていけそうな? ) 図形の上でだけ考えておいて, あとは線型に伸ばすということで全体の性質をみるということがあります. ここではそれを行なっています. 函数解析学などでは, 無限次元上に定まる理論上重要な写像( いわゆる有界線型作用素 )に対してノルムを定めますが, それらのノルムは, 無限次元の球面だけで見れば良いという性質があったりします. \par
最後に証明の中で出てきた $\| \cdot \| _ 1$ というノルムですが, これはマンハッタン距離と呼ばれる距離です. 解析などでは計算が楽に済むので利用されます. ( 他にも応用はいくつかあると思いますが, 僕は知りません笑)
\newpage
問5:\par 
皆さんには, 冗長な指摘かもしれませんが, これ, 
	\begin{itemize}
		\item $C ( K )$ が線型空間であること
		\item $\max \{ | f ( x ) \mid x \in K \}$ が存在していること
\end{itemize}
を確認しなければそもそもノルムを定義したことにならないです. 後輩とのゼミなどでこの辺りうっかりミスが多かったのでコメントしておきました. \\
\dotfill
\begin{leftbar}
	\begin{proof}
	連続函数の定数倍, 連続函数の和が再び連続函数になることから, $C ( K )$ は線型空間になる. また, $K$ がcompactであるから, $f ( K )$ はcompactである. また, $| \cdot |$ も連続函数であるから, $| f ( K ) |$ も compactである. 従って最
	大値を持つから, $\| f \|$ なる値が存在していることがわかる. ノルムの条件を満たすことを示す. 
		\begin{enumerate}
			\item $\| f \| \geq 0$ であること
				
				\parindent=1zw 各 $x \in K$ に対して, $| f ( x ) | \geq 0$ である. 従って, $\| f \| = \max \{ | f ( x ) \mid x \in K \} \geq 0$
				
			\item $\| f \| = 0 \Leftrightarrow f = 0$ であること ( $f = 0$ は写像として定数函数 $0$ に等しいという意味です. )	
				
				$f = 0$ であれば, 各 $x \in K$ に対して $| f ( x ) | = 0$ であるから, $\| f \| = 0$ は明らか. 逆を示す. $\| f \| = 0$ とする. 各 $x \in K$ に対して, $| f ( x ) | = 0$ である. つまり $f ( x ) = 0$ . よって $f = 0$
				
			\item $\| f + g \| \leq \| f \| + \| g \|$ であること
			
				\begin{eqnarray*}
					\| f + g \| & = & \max \{ | f ( x ) + g ( x ) \mid x \in K \} \\
					& \leq & \max \{ | f ( x ) | + | g ( x ) | \mid x \in K \} \\
					& = & \max \{ | f ( x ) |  \mid x \in K \} + \max \{ | g ( x ) |  \mid x \in K \} = \| f \| + \| g \|
				\end{eqnarray*}
		\end{enumerate}
	よって, 確かにノルムである. 
\end{proof}
\end{leftbar}
\newpage

問6:\\
\dotfill
\begin{leftbar}
	\begin{proof}
	収束先となる函数 $f$ を構成し, i) 収束先であること ii) 連続であること  を示す. 
	
	各 $x \in K$ において, 実数列 $( f_n ( x ) ) _ {n \in \mathbb N}$ はCauchy列である. 実際, 
%		
		\[
			| f _n ( x ) - f _m ( x ) | \leq \max \{ | f_n ( x ) - f_m ( x ) \mid x \in K \} = \| f_n - f_m \| \to 0 \ ( n , m \to N )
		\]
%		
	よりわかる. 実数の完備性から列 $( f_n ( x ) )$ は収束する. そこで, 
%		
		\[
			f : K \ni x \mapsto \lim_{n \to \infty} f_n ( x ) \in \mathbb R
		\]
%		
	と定める. この $f$ が i ) , ii ) を満たすことを示す. \\
	i) を満たすこと. すなわち, 
%		
		\[
			\forall \varepsilon > 0 \text {に対して, } \exists N \in \mathbb N \ \mathrm{s.t.} \ \forall n > N \text {に対して, } \| f_n - f \| < \varepsilon
		\]
%
	を示す. ただし, $\| \cdot \|$ の定義から, $\| f_n - f \| < \varepsilon$ は, 
%		
		\[
			\forall t \in K \text {に対して, } | f_n ( t ) - f ( t ) | < \varepsilon
		\]
%		
	を示せばよいことに注意する. $( f_n )$ が一様ノルムに関してCauchyの収束条件を満たすことに $\varepsilon / 2 > 0$ を適用すると, ある自然数 $N_1$ が存在して, 
%		
		\[
			\forall n , m > N _1 \text {に対して, } \| f_n - f_m \| < \varepsilon / 2
		\]
%		
	を満たす. $N : = N_1$ とおく. $n > N$ , $t ¥in K$ とする. 列 $( f_n ( t ) )$ は $f (t)$ に収束するから, ある自然数 $N_2$ が存在して, 
%		
		\[
			\forall m > N_2 \text {に対して, }| | f_{m} ( t ) - f ( t ) | < \varepsilon / 2
		\] 
%		
	を満たす. そこで, $m > \max \{ N_1 , N_2 \}$ をとれば, 
%		
		\[
			| f_n ( t ) - f ( t ) | \leq | f_n ( t ) - f_m ( t ) | + | f _m ( t ) - f ( t ) | \leq \| f_n - f_m \| + | f_m ( t ) - f ( t ) | < \varepsilon
		\] 
%		
	となる. よって収束先である. \\
	
	ii) を満たすこと. すなわち, 各 $\alpha \in K$ において
%		
		\[
			\varepsilon > 0 ¥text {に対して, } \exists \delta > 0 \ \mathrm{s.t.} \ \forall x \in K \ \mathrm{s.t.} \ d ( \alpha , x ) < \delta \text {に対して, } | f ( x ) - f ( \alpha ) | < \varepsilon
		\]
%		
	を満たすことを示す. $\alpha \in K$ とする. $\varepsilon > 0$ とする. 列 $( f_n ( \alpha ) )$ が $f ( \alpha )$ に収束することから, ある自然数 $N_1$ が存在して, 
%		
		\[
			\forall n > N_1 \text {に対して, } | f_n ( \alpha ) - f ( \alpha ) | < \varepsilon / 3
		\]
%	
	が成り立つ. また, $( f_n )$ がCauchyの収束条件を満たすことから, ある自然数 $N_2$ が存在して, 
%		
		\[
			\forall n , m > N_2 \text {に対して, } \| f_n - f_m \| < \varepsilon / 3 
		\]
%		
	が成り立つ. $N : = \max \{ N_1 , N_2 \}$ とおく. $f_N$ は $K$ 上で連続であるから, $\alpha$ での連続性より, ある正数 $\delta'$ が存在して, 
%		
		\[
			\forall x \in K \ \mathrm{s.t.} \ d ( \alpha , x ) < \delta' \text {に対して, } | f_N ( x ) - f_ N( \alpha ) | < \varepsilon / 3
		\]
%	
	が成り立つ. $\delta : = \delta'$ とおく. $x \in K$ かつ $d ( \alpha , x ) < \delta$ とする. 
%		
		\[
			| f ( x ) - f ( \alpha ) | \leq | f ( x ) - f_N ( x ) | + | f_N ( x ) - f_N ( \alpha ) | + | f_N ( \alpha ) - f ( \alpha ) | \leq \| f - f_N \| + \varepsilon / 3 + \| f - f_N \| < \varepsilon
		\]
%		
	より $f$ は連続である. つまり $f \in C ( K )$ である. 
	
	\end{proof}
\end{leftbar}
これ, Compact上の連続函数であることはほとんど本質的なことに影響しません. 定義域がCompactである条件を外す代わりに, 有界連続函数に対して適用すれば同様の議論が成り立ちます. ( ただし, その場合, 最大値ではなく上限でノルムを定義することになります. ) もしくは, 定義域上でCompact - support ( あるCompact集合上で値をもち, それ以外では恒等的に0 ) など他にも少し条件を変えて修正することでいくつかの応用が存在します. また, 詳しく調べてないのでわかりませんが ,定義域が距離空間であることもあまり影響しなかったと思います. \par 
函数解析における函数空間の一例です. このような $C(K)$ に相当する空間として, 微分可能函数空間などもあげられます. ここは, 深入りすると, それだけで大学一年間分の解析の授業ができるぐらいですので, ここで止めておきます. \\
参考\par 
宮寺功 関数解析 ( ちくま学芸出版 )\par 
洲之内治男 関数解析入門 ( 近代ライブラリ社 )
\newpage
\section{微分法}

\newpage

\section{初等函数}

\newpage

\section{積分法}


\subsection{p239}

\begin{screen}
	p239 \, (i) \par 
	 $\tan \frac{x}{2}=t$とおくと,
	\begin{align*}
		\int_{0}^{\frac{\pi}{2}} \cfrac{\sin x}{1+\cos x} \, dx & = \int_{0}^{1} \cfrac{\cfrac{2t}{1+t^2}}{1+\cfrac{1-t^2}{1+t^2}} \cdot \cfrac{2}{1+t^2} \, dt \\
		& = \int_{0}^{1} \frac{2t}{1+t^2} \, dt \\
		& = \Bigl[\log (t^2+1)\Bigl]_{0}^{1} \\
		& = \log 2-0 = \log 2
	\end{align*}
\end{screen}

\begin{screen}
	p239 \, (ii) \par
	  $x-a=a \sin \theta$\,($ -\pi \le \theta < \pi$)とする置換を用いる.
	\begin{align*}
		\int_{0}^{a} \sqrt{2ax-x^2} \, dx & = \int_{0}^{a} \sqrt{-(x-a)^2+a^2} \, dx \\
		& = \int_{-\frac{\pi}{2}}^{0} a \sqrt{1-\sin ^2 \theta } \cdot a\cos \theta \, d \theta \\
		& = \int_{-\frac{\pi}{2}}^{0} a |\cos \theta| \cdot a\cos \theta \, d \theta \\
		& = \int_{-\frac{\pi}{2}}^{0} a^2 \cos^2 \theta \, d \theta \\
		& = \int_{-\frac{\pi}{2}}^{0} a^2 \left (\frac{1+\cos 2 \theta }{2}\right) \, d \theta \\
		& = \frac{1}{2} a^2 \left [\theta + \frac{1}{2}\sin 2 \theta \right ]_{-\frac{\pi}{2}}^{0} \\
		&= \frac{\pi a^2}{4}
	\end{align*}
\end{screen}

\begin{screen}
	p239 (iii) 
	\begin{align*}
		|\sin 2 \theta| =
		\begin{cases}
			\sin 2 \theta & (0 \le \theta < \frac{\pi}{2} のとき)\\
			- \sin 2 \theta & (\frac{\pi}{2}\le \theta \le \pi のとき)
		\end{cases}
	\end{align*}
		なので,
		\begin{align*}
			\int_{0}^{\pi} |\sin 2 \theta| \, d \theta & = \int_{0}^{\frac{\pi}{2}} \sin 2 \theta \, d \theta +\int_{\frac{\pi}{2}}^{\pi} (-\sin 2 \theta) \, d \theta \\
			&= \left [-\frac{\cos 2 \theta}{2}\right]_{0}^{\frac{\pi}{2}} + \left [\frac{\cos 2 \theta}{2}\right]_{\frac{\pi}{2}}^{\pi} \\
			& = -\frac{(-1-1)}{2} + \frac{1+1}{2} = 2
		\end{align*}
	\end{screen}

	\begin{screen}
		p239 (iv) 
		\begin{align*}
			\int_{0}^{\pi} e^{inx} \, dx  =
			\begin{cases}
				2 \pi & (n=0 のとき) \\
				0 & (n \in \mathbb{Z}\setminus \{0\} のとき)
			\end{cases}
		\end{align*}
		である.$n=0$のときは
		\begin{align*}
			\int_{0}^{2 \pi} e^{inx} \, dx & = \int_{0}^{2\pi} \, dx \\
			& = \Bigl[x\Bigl]_{0}^{2\pi} = 2\pi
		\end{align*}
		となり,$n \ne 0$のときは
		\begin{align*}
			\int_{0}^{2\pi} e^{inx} \, dx & = \left [\frac{e^{inx}}{in} \right ]_{0}^{2\pi} \\
			& = \frac{1}{in} (1-1)=0
		\end{align*}
		となる.
	\end{screen}

	\begin{screen}
		p239 (v) \par 
		$m=n$のとき,
		\begin{align*}
			\int_{0}^{2\pi} \cos m x \sin nx \, dx & = \int_{0}^{2\pi} \cos mx \sin mx \, dx \\
			& = \int_{0}^{2\pi} \left (\frac{\sin 2mx + \sin 0}{2}\right ) \, dx \\
			& = \left [-\frac{\cos 2mx}{2m}\right ]_{0}^{2\pi} =0
		\end{align*}
		となる.$m \ne n$のとき,
		\begin{align*}
			\int_{0}^{2\pi} \cos mx \sin nx \, dx & = \int_{0}^{2\pi} \left (\frac{\sin (m+n)x + \sin (n-m)x}{2}\right) \, dx \\
			& = \frac{1}{2}\left [\frac{\sin (m+n)x}{m+n}+\frac{\sin (n-m)x}{n-m} \right]_{0}^{2\pi} =0
		\end{align*}
		である,ここまでの議論と,$m$と$n$の対称性により,
			\[
				\int_{0}^{2\pi} \cos mx \sin nx \, dx =\int_{0}^{2\pi} \sin mx \cos nx \, dx =0 ~(n,m \in \mathbb{N})
			\]
		となる.
	\end{screen}

	\begin{screen}
		p239 (vi) \par 
		$m \ne n$のとき,
		\begin{align*}
			\int_{0}^{2\pi} \cos mx \cos nx \, dx & = \int_{0}^{2\pi} \left (\frac{\cos(m+n)x+\cos(n-m)x}{2}\right) \, dx\\
			& = \frac{1}{2} \left [\frac{\sin (m+n)x}{m+n}+\frac{\sin(n-m)x}{n-m}\right]_{0}^{2\pi} \\
			& = 0-0 =0
		\end{align*}
		となる.$m =n \ne 0$のとき,
		\begin{align*}
			\int_{0}^{2\pi} \cos mx \cos nx \, dx & = \int_{0}^{2\pi} \cos^2 mx \, dx \\
			& = \int_{0}^{2\pi} \left (\frac{1+\cos 2mx}{2}\right) \, dx  \\
			& = \left [\frac{x}{2}+\frac{\sin 2mx}{4m}\right]_{0}^{2\pi} = \pi
		\end{align*}
		となる.$m =n =0$のとき,
		\begin{align*}
			\int_{0}^{2\pi} \cos mx \cos nx \, dx & = \int_{0}^{2\pi} dx \\
			& = \Bigl[x\Bigl]_{0}^{2\pi} = 2\pi
		\end{align*}
		となる,以上をまとめて,
		\begin{align*}
			\int_{0}^{2\pi} \cos mx \cos nx \, dx =
			\begin{cases}
				0 & (m \ne n のとき)\\
				\pi & (m = n\ne 0のとき)\\
				2 \pi & (m=n=0 のとき)
			\end{cases}
		\end{align*}
		である.
	\end{screen}

\newpage

\section{級数}

\end{document}

